% Intended LaTeX compiler: pdflatex
\documentclass[11pt]{article}
\usepackage[utf8]{inputenc}
\usepackage[T1]{fontenc}
\usepackage{graphicx}
\usepackage{longtable}
\usepackage{wrapfig}
\usepackage{rotating}
\usepackage[normalem]{ulem}
\usepackage{amsmath}
\usepackage{amssymb}
\usepackage{capt-of}
\usepackage{hyperref}
\makeatletter
\newcommand{\citeprocitem}[2]{\hyper@linkstart{cite}{citeproc_bib_item_#1}#2\hyper@linkend}
\makeatother
\author{Konstantinos Vasilakos}
\date{\today}
\title{Lick the Toad\\\medskip
\large A web based interface for displaced agencies in musical interaction}
\hypersetup{
 pdfauthor={Konstantinos Vasilakos},
 pdftitle={Lick the Toad},
 pdfkeywords={},
 pdfsubject={},
 pdfcreator={Emacs 28.2 (Org mode 9.6)}, 
 pdflang={English}}
\begin{document}

\maketitle
\setcounter{tocdepth}{1}
\tableofcontents



\section*{Introduction}
\label{sec:org674f903}
\section*{Lick the Toad}
\label{sec:orgda53d64}
Ltt is a web interface designed for bridging the gap between performer(s) and audience, allowing former and latter to engage interactively.

\section*{Bridging the audience agency in musical interaction}
\label{sec:orge486533}
It allows for real time of data interaction between clients and server offering an ideal tool to share data for musical interpretation on the fly.
\subsection*{Back to the parlour}
\label{sec:orgfb38a5d}
\begin{quote}
\ldots{} as the schism between ‘popular’ and ‘art’ deepened, and
the latter demanded increasing levels of virtuosity in order
to realise musical ideas, performance of certain strands of
contemporary music became nearly impossible for anyone
but professionals; disappearing from the ‘soiree’ repertoire.
\end{quote}

\emph{Fischman, R. (2011). Back to the parlour. Sonic Ideas – Ideas Sónicas, 3(2): 53–66.}
\subsection*{Some implementation notes:}
\label{sec:org6d0662a}
\begin{enumerate}
\item Real time communication using web sockets\footnote{Web sockets is a real time communication mechanism that allow web pages to send and receive data amongst peers.

\label{bibliographystyle link}
\bibliographystyle{unsrtnat}

\label{bibliography link}}
\item Sonifications on users' devices.
\item OSC communication with: MaxMSP, SuperCollider, Pure Data, etc.
\end{enumerate}

\section*{Background}
\label{sec:org6619394}
\begin{center}
\includegraphics[width=.9\linewidth]{/Users/konstantinos/.emacs.d/.local/cache/org/persist/ec/6de2c0-fc8e-432e-b79d-c9d8965e7d45-7f1010a6ca72c3902a58faa6f9ff74b9.jpg}
\end{center}

\subsection*{Looking Back}
\label{sec:org5f416d2}
“The Art of and Apparatus for Generating and Distributing Music Electrically”.\footnote{A \href{https://120years.net/wordpress/the-telharmonium-thaddeus-cahill-usa-1897/}{new field of electronic musical instruments} and \href{https://artsandculture.google.com/story/iAWRKDY1jD1jKA}{electronic musical instruments creation using telegraphy}.} \ref{Table_Crab_Telharmonium}
\begin{table}[htbp]
\label{Table_Crab_Telharmonium}
\centering
\begin{tabular}{llr}
Name & Instrument & Year\\\empty
\hline
Puskás & Telefonhírmondó & 1893\\\empty
Cahill & Telharmonium & 1895/97\\\empty
Gray & Musical Telegraph & 1874\\\empty
Ader & Théâtrophone & 1881\\\empty
Soemering & n/a & 1809\\\empty
\end{tabular}
\end{table}

\emph{Crab, S. (2013). The ‘Telharmonium’ or ‘Dynamophone’ Thaddeus Cahill, USA 1897.}

\begin{NOTES}
In 1895 Thaddeus Cahill submitted his first patent for the Telharmonium
By Unknown author - The World's Work, June 1906, vol. XII, no. II, Public Domain, \url{https://commons.wikimedia.org/w/index.php?curid=112285881}
\end{NOTES}

\subsection*{Looking Forward}
\label{sec:orgf38e5a6}
Modern takes using telecom: the networked music paradigm \emph{Collins, N., \& Escrivan Rincón, J. d. (2011). The Cambridge companion to electronic music. Cambridge: Cambridge University Press.}

Previous works by the author in this field:
\begin{itemize}
\item \href{https://serkansevilgen.com/docs/01\_ICLC\_2021\_Sevilgen\_Vasilakos\_Wilson.pdf}{BEER Pea Stew: Recalibrated} (Wilson, Vasilakos, Lorway, Margetson, Yeung).
\item \href{https://serkansevilgen.com/docs/01\_ICLC\_2021\_Sevilgen\_Vasilakos\_Wilson.pdf}{ICE: Symphony in Blue 2.0} \textbf{based on Kamran Ince's initial work} (Vasilakos, Sevilgen, Dagdeviren, Wilson)
\end{itemize}

\section*{Ltt: machine Learning in the browser}
\label{sec:orga8d0465}
\subsection*{Client Collection Data}
\label{sec:org006e1b7}
A client allows to train a model in a web browser and triger predictions based on a regression algorithm. Once the training process is done the system offers visualization of the output values. The sound generation is based on a Markov chain module which is generating predictions based on the ml predictions. At the same time the client is able to send these data to other clients ``listening''  for incoming OSC messages.
\subsection*{Notes on ML in Computer Music and Digital Arts}
\label{sec:org1672f84}
For some machine learning in Electronic Music literature as follows:
\begin{itemize}
\item Vasilakos, K. (2022). A Networked Hybrid Interface for Audience Sonification and Machine Learning. Revista Vórtex, 10(1) \url{http://dx.doi.org/10.33871/23179937.2022.10.1.4695}
\item Collins, N. (2015). Live Coding and Machine Listening. In Proceedings of the First International Conference on Live Coding (pp. 8). Leeds, UK.
\item Fiebrink, R., \& Sonami, L. (2020). Reflections on Eight Years of Instrument Creation with Machine Learning. In R. Michon, \& F. Schroeder, Proceedings of the International Conference on New Interfaces for Musical Expression (pp. 282–288). Birmingham, UK: Birmingham City University.
\item Baalman, M. (2020). The machine is learning.
\item Amershi, S., Cakmak, M., Knox, W. B., \& Kulesza, T. (2014). Power to the People: The Role of Humans in Interactive Machine Learning. AI Magazine, 35(4), 105–120. \url{http://dx.doi.org/10.1609/aimag.v35i4.2513}
\end{itemize}

\section*{Ltt's creative directions}
\label{sec:org9aef2df}
\begin{NOTES}
Ltt has been used as a standalone app for collective sonifications of bystanders and remote participants but since then it has taken many spins including a real time chat engine amongst peers.
\end{NOTES}

\subsection*{Live Coding: what now?}
\label{sec:org5e6861f}
In live coding performances, there is always the question of how a coder is taking their decisions while changing the code on the fly, described also as ``kairotic coding''. \emph{Cocker, E. (2018). What now, what next — kairotic coding and the unfolding future seized. Digital Creativity, 29(1), 82–95. \url{http://dx.doi.org/10.1080/14626268.2017.1419978}}

\subsubsection*{Ongoing Work}
\label{sec:org41313e7}
A study on live coding using ltt and stochastic processes\footnote{Stochastic synthesis is coined by Iannis Xenakis, with some of the most famous works Legend Air and Gendy system.} in SuperCollider's JITLib.
\begin{verbatim}
{
	[nil].choose;
}
\end{verbatim}

\href{https://youtube.com/watch?v=IrGk0yrfbOY}{[nil]​}

\section*{Discussion (instead of) Conclusion}
\label{sec:orgad60544}
While ltt serves both as a standalone and live coding tool, it arguably allows for a greater coherence amongst peers on live performance. Similar to the concept of ``back to the parlour'' where members of the audience are able to enact an impromptu improvisation.

\section*{Thanks}
\label{sec:org96c2b98}
Courtesy to the majestic Org mode
\begin{center}
\includesvg[width=.9\linewidth]{/Users/konstantinos/.emacs.d/.local/cache/org/persist/60/c86ac0-aa70-4345-b2b7-a3aa1b592a75-b12c0615bd9cca27d21e0c1e4079751a}
\end{center}

\ldots{}and the powerful SuperCollider
\begin{center}
\includesvg[width=.9\linewidth]{/Users/konstantinos/.emacs.d/.local/cache/org/persist/82/a01565-1a8b-4360-a7a0-fe1130d7d9bf-d3955a7c00c4fd105e49e34b7c2707cf}
\end{center}
\end{document}